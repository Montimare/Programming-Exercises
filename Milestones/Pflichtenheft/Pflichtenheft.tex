\documentclass[]{article}

%opening
\usepackage{hyperref}
\hypersetup{
	colorlinks=true,
	linktoc=all,
	linkcolor=black
}
%Fehlt: 1, 2, 3, 4, 5, 6, 7, 8, 9, 12, 13, 14
\begin{document}
	\begin{titlepage}
		\centering
		\vspace*{0cm}
		{\scshape\Large Frankfurt University of Applied Sciences}\\[3cm]
		{\huge\bfseries TeamCalendar}\\[8cm]
		{\Large\itshape Team ProgExTRAORDINAIRE:}\\
		{\Large\itshape Klejdi Galushi, Marc Roemer, Felix Schneider}\\[3cm]
		Supervisor:\\
		Salvatore Sabba\\[1cm]
		{\large \today}
	\end{titlepage}
	\newpage
	\tableofcontents
	\newpage
	
\section{Target determination}
	\subsection{Mandatory criteria}
		\begin{itemize}
			\item Big readable text
		\end{itemize}
	\subsection{Target criteria}
	\subsection{Optional criteria}
	\subsection{Exclusion criteria}
\section{Product deployment}
	\subsection{Application area}
	\subsection{Target groups}
		The target groups for this tool are:
		\begin{itemize}
			\item Students
			\item Companies
			\item Sport clubs
			\item Institutions
		\end{itemize}
		Based on this it is assumed that the frontend needs to be readable by people of all age groups.
	\subsection{Operating conditions}
	\subsection{Development setup}
		\begin{enumerate}
			\item Create a venv in \texttt{./Backend} using \texttt{requirements.txt} with \texttt{pip install --upgrade -r requirements.txt}
			\item Activate venv
			\item Navigate to \texttt{mysite} and run development server with \texttt{python manage.py runserver}
			\item Execute \texttt{npm install} in \texttt{./Frontend}
			\item Navigate to \texttt{my-app} and run development server with \texttt{npm start}
		\end{enumerate}
\section{Product overview}
\section{Product functions}
\section{Product data}
	To be saved Data will include:
	\begin{itemize}
		\item Username
		\item Calendar events consisting of:
			\subitem Title
			\subitem Start Time
			\subitem End Time
			\subitem Reminder Time
		\item Users with access to calendar
	\end{itemize}
\section{Product services}
\section{Quality requirements}
\section{User interface}
\section{Non-functional requirements}
\section{Technical product environment}
	\subsection{Software}
		\subsubsection{Client}
			\begin{itemize}
				\item Any browser that supports react 18.3.0 web apps or newer
			\end{itemize}
		\subsubsection{Server}
			\begin{itemize}
				\item python (\texttt{3.12.2})
				\item Django (\texttt{5.0.4})
				\item django-filter (\texttt{24.2})
				\item djangorestframework (\texttt{3.15.1})
				\item SQLAlchemy (\texttt{2.0.30})
				\item MySQL
			\end{itemize}
	\subsection{Hardware}
		\subsubsection{Server}
			Linux or Windows server with internet access.\\
			Mac not tested.
		\subsubsection{Client}
			Any device capable of running a browser that supports react 18.3.0 web apps or newer with internet access.
	\subsection{Orgware}
		The deployed code needs to be documented to make servicing it possible. Intended documentation is the GitHub project where developers made comments for their decisions, Mask Draft and an Entity-relationship diagram for the database.
	\subsection{Product interfaces}
		The frontend and backend need to be connected via their ip adresses in a network.\\
		The backend and database need to be connected via their ip adresses in a network.\\
		External interfaces are not needed.
\newpage
\section{Special requirements for the development environment}
	\subsection{Software}
		\begin{itemize}
			\item Git for downloading the project
			\item GitHub for accessing the project
			\item PyCharm for editing the backend and database
			\item Visual Studio Code for editing the frontend and backend
			\item python (\texttt{3.12.2})
			\item nodejs (\texttt{20.12.2})
			\item @fullcalendar/core (\texttt{6.1.11 or newer}) 
			\item @fullcalendar/daygrid (\texttt{6.1.11 or newer}) 
			\item @fullcalendar/interaction (\texttt{6.1.11 or newer}) 
			\item @fullcalendar/react (\texttt{6.1.11 or newer}) 
			\item @testing-library/jest-dom (\texttt{5.17.0 or newer}) 
			\item @testing-library/react (\texttt{13.4.0 or newer}) 
			\item @testing-library/user-event (\texttt{13.5.0 or newer}) 
			\item axios (\texttt{1.6.8 or newer}) 
			\item react (\texttt{18.3.0 or newer}) 
			\item react-dom (\texttt{18.3.0 or newer}) 
			\item react-scripts (\texttt{5.0.1 or newer}) 
			\item web-vitals (\texttt{2.1.4 or newer})
			\item asgiref (\texttt{3.8.1})
			\item Django (\texttt{5.0.4})
			\item django-filter (\texttt{24.2})
			\item djangorestframework (\texttt{3.15.1})
			\item Markdown (\texttt{3.6})
			\item sqlparse (\texttt{0.5.0})
			\item SQLAlchemy (\texttt{2.0.30})
			\item MySQL
		\end{itemize}
	\subsection{Hardware}
		Windows or Linux computer capable of running all software outlined in 11.1 Software.\\
		Mac not tested.
	\subsection{Orgware}
		The agile software development method will be used. GitHub will provide all tools necessary for agile development. Sprints last a week and end on Thursdays in the lecture.\\
		\\
		GitHub is used for:
		\begin{itemize}
			\item Storing code
			\item Managing code
			\item Sharing code
			\item Version control
			\item Continuous Integration/Testing (GitHub actions)
			\item Access control
			\item Bug tracking (Issues)
			\item Task management (GitHub projects)
		\end{itemize}
	\subsection{Development interfaces}
%eventuell noch ausformulieren wie genau das geht?-----------------------------------------------------------------------------------------------------------------------------------------------------------------------------------------------------------------------------
		The frontend and backend need to be connected via their ip adresses in a development environment.\\
		The backend and database need to be connected via their ip adresses in a development environment.\\
		External development interfaces are not needed.
\section{Division into sub-products}
\section{Additions}
\section{Glossary}
\section{Sources}
	\begin{itemize}
		\item https://en.wikipedia.org/wiki/GitHub | viewed: 10.06.2024
	\end{itemize}


\end{document}
